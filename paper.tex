\documentclass{article}
\title{The Epistemic Modal \emph{probable}: Qualitative and Quantitative Accounts}
\author{Conrad Friedrich \\ MCMP, LMU Munich \\ 1,337 words}
\date{\today} 

\usepackage[utf8]{inputenc}
\usepackage{amsmath}
\usepackage{amssymb}
\usepackage{amsthm}
\usepackage{stmaryrd}
\usepackage{enumitem}
\usepackage{setspace}
% Used to scale \medcap
\usepackage{graphicx}
\usepackage{ebproof/ebproof}
\usepackage{multicol}

\usepackage[backend=biber,authordate,
ibidtracker=context,natbib,doi=false,isbn=false,url=false]{biblatex-chicago}

\addbibresource{/home/gilbert/Documents/bibliography/references.bib}

\newcommand*{\medcap}{\mathbin{\scalebox{1.5}{\ensuremath{\cap}}}}%
\renewcommand{\L}{\mathcal{L}}
\newcommand{\lb}{\llbracket}
\newcommand{\rb}{\rrbracket}
\begin{document}

\onehalfspacing
\maketitle

\section{Introduction}


\section{Intensional Semantics}
Note that we don't need first order logic. 
Abstracted away from syntax, just look at the semantics.
NEAT summary in Ivano's script, Class 9, page 3/4
\section{The Epistemic Modal \emph{probable}}
Also show the need for expanding the standard intensional semantics!
How is probable to be evaluated? MacFarlane's Solipcism. 

To base the following discussions on a common language, we follow the definitions of a propositional language $\L$ by \textcite{harrison-trainor17_prefer} and \textcite{holliday13_measur}:
\[
\varphi ::= p ~|~ \neg \varphi ~|~ (\varphi \land \psi) ~|~ (\varphi \rightarrow \psi) ~|~ (\varphi \vee \psi) ~|~ (\varphi \succeq \psi) ~|~ \triangle\varphi 
\]
The intended readings for the Boolean operators are completely standard.
\begin{align*}
  \varphi \succeq \psi &: \varphi \text{ is at least as likely as } \psi \\
  \triangle \varphi &:\text{probably } \varphi.
\end{align*}
We might leave uninterpreted some elements which don't figure prominently in the respective theory discussed at that moment.

\section{Kratzer's Analysis}

For \textcite{kratzer91_modal} epistemic modals like \emph{probably} are just a special case of a more general class of modal auxiliaries.
These can be epistemic (e.g. `Jockl must have been the murderer [in view of the available evidence]'), deontic (e.g. `Jockl must go to jail [in view of what the law provides]'), or circumstancial (e.g. `Jockl can lift the rock [given the weight of the rock and the conditions of Jockl's muscles, etc.]').
Further, there are teleological (in view of certain aims), bouletic (in view of certain wishes), and doxastic (in view of certain beliefs) modalities.
This list is by no means complete.

Kratzer postulates that modality is always relative to a certain kind of interpreting the modalities involved, following Peirce (cite).
She set out to formulate these different modalities in a common framework such that we'd only need to specify some parameters of the framework to determine the relevant modalities and give truth conditions for sentences involving modalities, while focussing on epistemic modals.

Kratzer is unhappy with what she calls the \emph{standard analysis} as is gives the wrong verdict in two salient scenarios.
--maybe also go through kratzer's examples which further motivate her decisions--maybe explain why we don't have an even easier definition.

Importantly, the account for modality needs to include graded modalities, of the following sort:
\begin{enumerate}[nosep,label=\alph*)]
  \item Michl must be the murderer.
  \item Michl is probably the murderer.
  \item There is a good possibility that Michl is the murderer.
  \item Michl might be the murderer.
\end{enumerate}
Here, the modalities are listed in decreasing strength, such that the intended reading is that a) implies b), b) implies c), and so forth. Note that is a qualitative approach to graded modals, as opposed to quantitative approaches, employing a scale structure in the semantics.
\textcite{yalcin10_probab_operat} calls it a `relative likelihood' approach.

\subsection{Kratzer's Formal Setup}
How do the semantics look like?
we assume a set $W$ of possible worlds, a set of propositions $P$ such that elements of $P$ are subsets of $W$, and assume that a proposition $p$ is true at $w$ iff $w \in p$.
The crucial element in Kratzer's semantics that distinguishes her account from most simple semantics via Kripke frames with an accessibility relation is the \emph{conversational background}.
It has two components.

The \emph{modal base} determines for every world the set of worlds which are epistemically accessible from it.
More formally, the modal base is a function $f: W \rightarrow \wp(P)$ assigning each world $w$ a set of propositions.
The set of propositions $f(w)$ assigned to $w$ represents, in the epistemic case, what we know at $w$.
The set of epistemically accessible worlds then is $\bigcap f(w)$.

The \emph{ordering source} is a function $g: W \rightarrow \wp(P)$ s.t. each $w$ is assigned a set of propositions $g(w)$ which interpreted as an \emph{ideal} (cf. \cite{lewis81_order_seman_premis_seman_count}). The ordering source induces a partial order (reflexive, antisymmetric and transitive) $\leq_{g(w)}$ as follows for all $w,w' \in W$: 
\[
w \leq_{g(w)} w' :\iff \{p: p\in g(w) ~\&~ w' \in p\} \subseteq \{p: p\in  g(w) ~\&~ w \in p\}
\]
The intended reading is that $w$ is at least as close to the ideal represented by $A$ as $w'$ iff all the ideal propositions in $A$ which are true in $w'$ are also true in $w$.
From this, Kratzer defines a the semantics for a list of different expressions, like `good possibility', `possibility' and so on.
For our purposes, it suffices to look at those definitions involved in defining the meaning of \emph{probably $\alpha$}.

First, define `as least as good a possibility as'.
The induced order on worlds $\leq_{g(w)}$ is `lifted' to an order on the level of propositions.
For a modal base $f$ and a ordering source $g$, for propositions $p,q$ and world $w$, we can define
\[
    p \succeq_w q :\iff \forall b \in B_w \exists a \in A_w: a \succeq_{g(w)} b,
\]
where $A_w = \bigcap f(w) \cap A$, and $B_w$ analogously (cf. \cite[][p.~519]{holliday13_measur}).
In words, a proposition $p$ is as least as good a possibility as a proposition $q$ seen from a world $w$ iff for any world in $q$ epistemically accessible from $w$ there is a world in $p$ accessible from $w$ that is at least as close to the ideal picked out $g(w)$.

Next, `a better possibility': $p \succ_w q :\iff p \succeq_w q ~\&~ q \not\succeq_w p$.  
Finally, we can assign a meaning to \emph{probably $\alpha$} given a set of worlds $W$, a modal base $f$ and a ordering source $g$ \parencite[][p.~645]{kratzer91_modal}:
\[
\llbracket \text{probably } \alpha \rrbracket^{f,g,w} = 1 :\iff \lb \alpha \rb \succ_w \lb \neg\alpha \rb
\]
Assuming here that we are looking at a flat fragment and $\alpha$ doesn't contain modals itself.
Intuitively,... [briefly explain]
\subsection{Discussion}

The account presented has a couple of advantages.

- Neatly fits with the other modals, just change f and g

- Gives an account of probable without resorting to probability measures (give argument by Hamblin - maybe more in philosophy texts? check titelbaum)

- solves problems describes in Kratzer text (probably dont reiterate)  

- more

But it also suffers from crippling counterexamples, as will be explained in the next section.

\section{Yalcin and Lassiter's Probability Semantics}
Introduce the adequacy conditions V1-VXX etc.

In his \parencite*{yalcin10_probab_operat}, Yalcin critically evaluates Kratzer's relative likelihood account and also Hamblin's account, which is based on the notion of possiblity (resp. plausiblity) measures \parencite{hamblin59_modal_probab}.
Yalcin does this by following the good philosophical practice of establishing a set of desiderata the accounts can be evaluated against. In the case for judging semantics for a natural language phenomenon, these naturally come in the form of inference patterns that strike one as intuitively valid or invalid.
The practice is fruitful because it gives criteria to compare different accounts in a transparent and comprehensible fashion, and shifts the focus of the problematic notion of \emph{intuitive plausibility} from judging the accounts themselves to judging the plausibility of the inference patterns.
Whether some formally specified semantics validates a given inference pattern is then simply not a matter of debate and can be shown mathematically given the assumptions one is forced to make explicit.   
\begin{center}
    \begin{prooftree}
        \hypo{ \triangle \varphi } \infer1[(V1)]{ \neg \triangle \neg \varphi}
    \end{prooftree}
\end{center}

\begin{center}
\begin{prooftree}
        \hypo{ \triangle (\varphi \land \psi)} \infer1[(V2)]{ \triangle \varphi \land \triangle \psi}
    \end{prooftree}
\end{center}

\begin{center}
\begin{prooftree}
        \hypo{ \triangle \varphi} \infer1[(V3)]{ \triangle (\varphi \lor \triangle \psi}
    \end{prooftree}
\end{center}

\begin{center}
\begin{prooftree}
    \hypo{ } \infer1[(V4)]{ \varphi \succeq \bot}
    \end{prooftree}
\end{center}

\begin{center}
\begin{prooftree}
    \hypo{ } \infer1[(V5)]{ \top \succeq \varphi}
    \end{prooftree}
\end{center}

\begin{center}
\begin{prooftree}
    \hypo{ \Box \varphi } \infer1[(V6)]{ \triangle \varphi}
    \end{prooftree}
\end{center}

\begin{center}
\begin{prooftree}
    \hypo{ \triangle \varphi } \infer1[(V7)]{ \Diamond \varphi}
    \end{prooftree}
\end{center}

\begin{center}
\begin{prooftree}
    \hypo{ \psi \succeq \varphi } \hypo{\triangle \varphi} \infer2[(V11)]{ \triangle \psi}
    \end{prooftree}
\end{center}

\begin{center}
\begin{prooftree}
    \hypo{ \psi \succeq \varphi } \hypo{\triangle \varphi} \infer2[(V8)]{ \triangle \psi}
    \end{prooftree}
\end{center}

\begin{center}
\begin{prooftree}
    \hypo{ \psi \succeq \varphi } \hypo{\varphi \succeq \neg\varphi}\infer2[(V12)]{ \psi \succeq \neg\psi}
    \end{prooftree}
\end{center}

\begin{center}
\begin{prooftree}
    \hypo{ \varphi \succeq \psi } \hypo{\varphi \succeq \chi}\infer2[(I1)]{ \varphi \succeq (\psi \lor \chd)}
    \end{prooftree}
\end{center}

\begin{center}
\begin{prooftree}
    \hypo{ \varphi \succeq \neg \varphi } \infer1[(I2)]{ \varphi \succeq \psi}
    \end{prooftree}
\end{center}

\begin{center}
\begin{prooftree}
    \hypo{ \triangle \varphi } \infer1[(I3)]{ \varphi \succeq \psi}
    \end{prooftree}
\end{center}
 
\section{Holliday and Icard's Generalisation to Measure Semantics}
\section{Holliday and Icard's Unification of Preferential and Probabilistic Semantics}
\section{Conclusion}
What about the $\epsilon$-semantics (Adams, Pearl)? 
\nocite{hamblin59_modal_probab,holliday13_measur,harrison-trainor17_prefer,kratzer91_modal,lassiter10_gradab,yalcin10_probab_operat,kratzer98_seman}
 \printbibliography
\end{document}