\documentclass{article}
\usepackage[utf8]{inputenc}

% \usepackage{aaai}
% \usepackage{times}
% \usepackage{helvet}
% \usepackage{courier}

\usepackage{amsmath}
\usepackage{amssymb}
\usepackage{amsthm}
\usepackage{stmaryrd}
\usepackage{enumitem}
\usepackage{setspace}

\usepackage{ebproof/ebproof}
\usepackage{calrsfs}
% Used for todo notes
\usepackage{xcolor}

\usepackage[backend=biber,authordate,
ibidtracker=context,natbib,doi=false,isbn=false,url=false]{biblatex-chicago}

\addbibresource{/home/gilbert/Documents/bibliography/references.bib}
\theoremstyle{definition}

\DeclareMathAlphabet{\pazocal}{OMS}{zplm}{m}{n}

\newtheorem{definition}{Definition}

\newcommand\todo[1]{\textcolor{red}{#1}}

\renewcommand{\L}{\pazocal{L}}
\newcommand{\F}{\pazocal{F}}
\newcommand{\lb}{\llbracket}
\newcommand{\rb}{\rrbracket}
\newcommand{\Po}{\text{Po}}

\newcommand{\succeqK}{\succeq^K_{M,w}}
\newcommand\wordcount{\documentclass{article}
\title{The Epistemic Modal \emph{probable}: Qualitative and Quantitative Accounts}
\author{Conrad Friedrich \\ MCMP, LMU Munich \\ 1,337 words}
\date{\today} 

\usepackage[utf8]{inputenc}
\usepackage{amsmath}
\usepackage{amssymb}
\usepackage{stmaryrd}
\usepackage{enumitem}
\usepackage{setspace}
% Used to scale \medcap
\usepackage{graphicx}
\usepackage[backend=biber,authordate,
ibidtracker=context,natbib,doi=false,isbn=false,url=false]{biblatex-chicago}

\addbibresource{/home/gilbert/Documents/bibliography/references.bib}

\newcommand*{\medcap}{\mathbin{\scalebox{1.5}{\ensuremath{\cap}}}}%
\renewcommand{\L}{\mathcal{L}}
\begin{document}

\onehalfspacing
\maketitle

\section{Introduction}
\section{Intensional Semantics}
Note that we don't need first order logic. 
Abstracted away from syntax, just look at the semantics.
\section{The Epistemic Modal \emph{probable}}
How is probable to be evaluated? MacFarlane's Solipcism. 

To base the following discussions on a common language, we follow the definitions of a propositional language $\L$ by \textcite{harrison-trainor17_prefer} and \textcite{holliday13_measur}:
\[
\varphi ::= p ~|~ \neg \varphi ~|~ (\varphi \land \psi) ~|~ (\varphi \rightarrow \psi) ~|~ (\varphi \vee \psi) ~|~ (\varphi \succeq \psi) ~|~ \triangle\varphi 
\]
The intended readings for the Boolean operators are completely standard.
$\varphi \succeq \psi$ is to be read as `$\varphi$ is at least as likely as $\psi$', and $\triangle \varphi$ is to be read as `probably $\varphi$'.

We might leave uninterpreted some elements which don't figure prominently in the respective theory discussed at that moment.

\section{Kratzer's Analysis}

For \textcite{kratzer91_modal} epistemic modals like \emph{probably} are just a special case of a more general class of modal auxiliaries.
These can be epistemic (e.g. `Jockl must have been the murderer [in view of the available evidence]'), deontic (e.g. `Jockl must go to jail [in view of what the law provides]'), or circumstancial (e.g. `Jockl can lift the rock [given the weight of the rock and the conditions of Jockl's muscles, etc.]').
Further, there are teleological (in view of certain aims), bouletic (in view of certain wishes), and doxastic (in view of certain beliefs) modalities.
This list is by no means complete.

Kratzer postulates that modality is always relative to a certain kind of interpreting the modalities involved, following Peirce (cite).
She set out to formulate these different modalities in a common framework such that we'd only need to specify some parameters of the framework to determine the relevant modalities and give truth conditions for sentences involving modalities, while focussing on epistemic modals.

Kratzer is unhappy with what she calls the \emph{standard analysis} as is gives the wrong verdict in two salient scenarios.
--maybe also go through kratzer's examples which further motivate her decisions--maybe explain why we don't have an even easier definition.

Importantly, the account for modality needs to include graded modalities, of the following sort:
\begin{enumerate}[nosep,label=\alph*)]
  \item Michl must be the murderer.
  \item Michl is probably the murderer.
  \item There is a good possibility that Michl is the murderer.
  \item Michl might be the murderer.
\end{enumerate}
Here, the modalities are listed in decreasing strength, such that the intended reading is that a) implies b), b) implies c), and so forth. Note that is a qualitative approach to graded modals, as opposed to quantitative approaches, employing a scale structure in the semantics. 

How does the semantics look like? For the basics: we assume a set $W$ of possible worlds, a set of propositions $P$ such that elements of $P$ are subsets of $W$, and that a proposition $p$ is true at $w$ iff $w \in p$. The crucial element in Kratzer's semantics is the \emph{conversational background}. It has two components.

The \emph{modal base} determines for every world the set of worlds which are epistemically accessible from it.
More formally, the modal base is a function $f: W \rightarrow \mathcal{P}(P)$\footnote{For finite $W$.} assigning each world $w$ a set of propositions.
The set of propositions $f(w)$ assigned to $w$ represents, in the epistemic case, what we know at $w$.
The set of epistemically accessible worlds then is $\bigcap f(w)$.

The \emph{ordering source} is a function $g: W \rightarrow \mathcal{P}(P)$ s.t. each $w$ is assigned a set of propositions $g(w)$ which interpreted as an \emph{ideal} (cf. \cite{lewis81_order_seman_premis_seman_count}). The ordering source induces an order $\leq_{g(w)}$ as follows for all $w,w' \in W$: 
\[
w \leq_A w' :\iff \{p: p\in A ~\&~ w' \in p\} \subseteq \{p: p\in A ~\&~ w \in p\}
\]
The intended reading is that $w$ is at least as close to the ideal represented by $A$ as $w'$ iff all the ideal propositions in $A$ which are true in $w'$ are also true in $w$. 

From this, Kratzer defines a the semantics for a list of different expressions, like `good possibility', `possibility' and so on. For our purposes, it suffices to look at those definitions involved in defining the model `probable'.
\begin{align*}
 \llbracket \varphi \succeq \psi \rrbracket^{f,g,w} &= \forall u \in (\medcap f(w) \cap q) \exists v \in \medcapf(w): v \leq_{g(w)}u ~\&~ v \in \llbracket \varphi \rrbracket 
\end{align*}



\section{Yalcin and Lassiter's Probability Semantics}
Introduce the adequacy conditions V1-VXX etc.
\section{Holliday and Icard's Generalisation to Measure Semantics}
\section{Holliday and Icard's Unification of Preferential and Probabilistic Semantics}
\section{Conclusion}
What about the $\epsilon$-semantics (Adams, Pearl)? 
\nocite{hamblin59_modal_probab,holliday13_measur,harrison-trainor17_prefer,kratzer91_modal,lassiter10_gradab,yalcin10_probab_operat,kratzer98_seman}
 \printbibliography
\end{document}} 

\title{The Epistemic Modal \emph{probable}}
\author{Conrad Friedrich \\ MCMP, LMU Munich \\ \wordcount words}
\date{\today} 


\begin{document}

\onehalfspacing

\maketitle

\tableofcontents
\section{Introduction}

\todo{
  \begin{itemize}[nosep]
    \item Write an Introduction
     \item be perfectly clear as to what happens. SIGNPOST 
  \end{itemize}
}

\subsection{Intensional Semantics}
In this paper, we will not be concerned with type theoretical considerations. It seems also that the relevant literature is not very keen on discussion that. One reason might be that it is rather simple. We just view the \emph{probable}-operator as an operator on sentences, such that we just add something along the lines of `if $\varphi:t$ then $\triangle \varphi:t$' to our type theory. The interesting developments for the purposes of this paper take place when we enrich the model for our calculus. As a further simplification, we will only be concerned with a simple propositional language to not add further complexity by considering first order models, as will be described in the next section.

\section{The Epistemic Modal \emph{probable}}
\todo{\begin{itemize}[nosep]
    \item  Give some natural language motivation
    \item Also show the need for expanding the standard intensional semantics!
    \item How is probable to be evaluated? Epistemic interpretation? MacFarlane's Solipcism.
    \item Mention how Lassiter has a different, graded adjective based approach.
    \item Discuss how absolute probability is best reduced to relative likelihood with the negation
      \item note that there are a lot more interesting problems associated with probable but to have an readable paper need to focus on particulars.
  \end{itemize}
}

To base the following discussions on a common language, we define a
propositional language $\L$ like \textcite{harrison-trainor17_prefer} and
\textcite{holliday13_measur} do:
\[
\varphi ::= p ~|~ \neg \varphi ~|~ (\varphi \land \psi) ~|~  (\varphi \succeq \psi) ~|~ \triangle\varphi 
\]
Where $p$ is an atomic formula, i.e. $p \in \text{Atom}$. The other boolean connectives $(\varphi \rightarrow \psi), (\varphi \vee
\psi)$ as well as $\top, \bot$ are defined in the standard way from negation and conjunction.
The intended readings are as follows:
\begin{align*}
  \varphi \succeq \psi &: \varphi \text{ is at least as likely as } \psi. \\
  \triangle \varphi &:\text{probably } \varphi. \\ 
\end{align*}

\subsection{Criteria for Adequacy}
In his \parencite*{yalcin10_probab_operat}, Yalcin critically evaluates Kratzer's relative likelihood account and also Hamblin's account, which is based on the notion of possiblity (resp. plausiblity) functions \parencite{hamblin59_modal_probab}.
Yalcin does this by following the good philosophical practice of establishing a set of desiderata the accounts can be evaluated against. In the case for judging semantics for a natural language phenomenon, these naturally come in the form of inference patterns that strike one as intuitively valid or invalid.
The practice is fruitful because it gives criteria to compare different accounts in a transparent and comprehensible fashion, and shifts the focus of the problematic notion of \emph{intuitive plausibility} from judging the accounts themselves to judging the plausibility of the inference patterns.
Whether some formally specified semantics validates a given inference pattern is then simply not a matter of debate and can be shown mathematically given the assumptions one is forced to make explicit.   
The rules are not meant to characterize a logic, of course.

\todo{
\begin{itemize}[nosep]
  \item Motivate and describe the conditions
  \item Describe what inference means (truth preservation)
\end{itemize}
}
\begin{center}
    \begin{prooftree}
        \hypo{ \triangle \varphi } \infer1[(V1)]{ \neg \triangle \neg \varphi}
    \end{prooftree}
\end{center}

\begin{center}
\begin{prooftree}
        \hypo{ \triangle (\varphi \land \psi)} \infer1[(V2)]{ \triangle \varphi \land \triangle \psi}
    \end{prooftree}
\end{center}

\begin{center}
\begin{prooftree}
        \hypo{ \triangle \varphi} \infer1[(V3)]{ \triangle (\varphi \lor \psi)}
    \end{prooftree}
\end{center}

\begin{center}
\begin{prooftree}
    \hypo{ } \infer1[(V4)]{ \varphi \succeq \bot}
    \end{prooftree}
\end{center}

\begin{center}
\begin{prooftree}
    \hypo{ } \infer1[(V5)]{ \top \succeq \varphi}
    \end{prooftree}
\end{center}

\begin{center}
\begin{prooftree}
    \hypo{ \psi \succeq \varphi } \hypo{\triangle \varphi} \infer2[(V11)]{ \triangle \psi}
    \end{prooftree}
\end{center}

\begin{center}
\begin{prooftree}
    \hypo{ \psi \succeq \varphi } \hypo{\varphi \succeq \neg\varphi}\infer2[(V12)]{ \psi \succeq \neg\psi}
    \end{prooftree}
\end{center}

\begin{center}
\begin{prooftree}
    \hypo{ \varphi \succeq \psi } \hypo{\varphi \succeq \chi}\infer2[(I1)]{ \varphi \succeq (\psi \lor \chi)}
    \end{prooftree}
\end{center}

\begin{center}
\begin{prooftree}
    \hypo{ \varphi \succeq \neg \varphi } \hypo{ \neg \varphi \succeq \varphi}\infer2[(I2)]{ \varphi \succeq \psi}
    \end{prooftree}
\end{center}

\begin{center}
\begin{prooftree}
    \hypo{ \triangle \varphi } \infer1[(I3)]{ \varphi \succeq \psi}
    \end{prooftree}
\end{center}

\subsection{Criteria not Considered}
Yalcin and also \textcite{lassiter10_gradab} introduce an additional criterion motivated by examples of the following sort:
\begin{enumerate}[nosep]
  \item There is a 60\% probability that it is raining.
  \item $\varphi$ is twice as likely as $\psi$.
\end{enumerate}
Yalcin argues that sentences of this type are ``used and interpreted by speakers who are innocent of the mathematical theory of probability'' \parencite[][p.~923]{yalcin10_probab_operat} and that a semantics for probability operators in natural language must be able to account for them, while Lassiter merely states it as a problem for preferential semantics \parencite[][p.~210]{lassiter10_gradab}. Both share the implicit assumption that both the qualitative talk (`probably $\varphi$') and the quantitative talk like in (2) above need to be accounted for by the same semantics. Given the methodological practice of frequent appeal to intuition to support claims of this sort, the assumption can also be doubted on the same grounds. It is equally plausible that people usually employ different notions of \emph{probable} (or \emph{likely}) in the qualitative and the quantitative case. Qualitative talk is quotidian and ubiquitous, whereas quantitative talk implies the notion of a probability function. It might be the case that speakers use it `innocently' without recourse to probability theory, but it's a different question of whether they do so competently. We argue that they might as well don't. The result is that the criterion does not necessarily help us evaluate adequate semantics for likelihood talk, and thus will be excluded from the present considerations.

Yalcin also discuss criteria concerning the combination of epistemic modals and indicative conditionals (V8-V10, not listed here). While undoubtedly very interesting, also considering semantics for conditionals and integrating those with epistemic modal semantics would be beyond the scope of this short paper.

Also not considered in this paper are more advanced discussions of intuitive validity of \emph{probable}-judgments. One example is a lottery with 1000 possible tickets.
Suppose Bloggs has 420 tickets, and 580 other players have one ticket each \parencite[p.~931]{yalcin10_probab_operat}. Is the following true or false?
`Probably, Bloggs is the winner of the lottery'. Although the chance of Bloggs winning is less than him not winning, intuitions here are apparently mixed. Throughout this paper, however, we assume that `probably $\varphi$' is equivalent to `$\varphi$ is more probable than not', ignoring the problematic intuitions just described.

In addition, we weren't able to include formal considerations on how \emph{probable}-talk relates to other epistemic modals like \emph{might, must}, etc., due to time and place constraints, although this aspect is particularly lending itself to analysis and would be the first to consider in a longer paper.

\section{Kratzer's Preferential Semantics}
For \textcite{kratzer91_modal} epistemic modals like \emph{probably} are just a special case of a more general class of modal auxiliaries.
These can be epistemic (e.g. `Jockl must have been the murderer [in view of the available evidence]'), deontic (e.g. `Jockl must go to jail [in view of what the law provides]'), or circumstancial (e.g. `Jockl can lift the rock [given the weight of the rock and the conditions of Jockl's muscles, etc.]').
Further, there are teleological (in view of certain aims), bouletic (in view of certain wishes), and doxastic (in view of certain beliefs) modalities.
This list is by no means complete.

Kratzer postulates that modality is always relative to a certain kind of interpreting the modalities involved.
She sets out to formulate these different modalities in a common framework such that we'd only need to specify some parameters of the framework to determine the relevant modalities and give truth conditions for sentences involving modalities, while focusing on epistemic modals.

Kratzer is unhappy with what she calls the \emph{standard analysis} as it gives the wrong verdict in two salient scenarios.
These are concerned with conflicting rules of law and an a combination of indicative conditionals and modals, respectively.
We will not reiterate these scenarios here, but--spoiler alert--Kratzer's theory manages to do them justice.

More relevantly, the account for modality according to Kratzer needs to include graded modalities, of the following sort:
\begin{enumerate}[nosep,label=\alph*)]
  \item Michl must be the murderer.
  \item Michl is probably the murderer.
  \item There is a good possibility that Michl is the murderer.
  \item Michl might be the murderer.
\end{enumerate}
Here, the modalities are listed in decreasing strength, such that the intended reading is that a) implies b), b) implies c), and c) implies d).
Note that is a qualitative approach to graded modals, as opposed to quantitative approaches. Some of these employ a scale structure in the semantics, as described by \textcite{lassiter10_gradab}.

Kratzer intends her account to fit into a common framework for other gradable modals.
The idea is to just change the parameters $f,g$--to be introduced shortly--to represent the constraints that different modalities impose upon the semantics.
If successful, this idea would come out in favor of Kratzer's account, since it unifies multiple semantics into a single formal framework.
In addition, we'd have semantics for epistemic modals and \emph{probable}-talk that do not need to make recourse to probability theory.
Whether these two features actually do speak in favor of Kratzer's account shall not be evaluated in this paper and are just now mentioned.
We stick to the criteria defined in the beginning.

How do the semantics look like?
We assume a set $W$ of possible worlds, propositions are subsets of $W$, and assume that a proposition $p$ is true at $w$ iff $w \in p$.
The crucial element in Kratzer's semantics that distinguishes her account from most simple semantics via Kripke frames with only a single accessibility relation is the \emph{conversational background}.
It has two components.

The \emph{modal base} determines for every world the set of worlds which are epistemically accessible from it.
More formally, the modal base is a function $f: W \rightarrow \wp(W)$ assigning each world $w$ a set of worlds.\footnote{It is not perfectly clear from \textcite[][p.~644]{kratzer91_modal}, but it seems that the modal base in fact assigns a set of propositions to each world, so a set of sets of worlds. This is as far as we can tell equivalent for our purposes to the definition proposed here, if we make slight amendments.}
The set of worlds $f(w)$ assigned to $w$ is intended to represent what we know at $w$.

The \emph{ordering source} is a function $g: W \rightarrow \wp(W)$ s.t. each $w$ is assigned a set of propositions $g(w)$ which interpreted as an \emph{ideal} (cf. \cite{lewis81_order_seman_premis_seman_count}). The ordering source induces an ordering $\leq_{g(w)}$ on $W$ for all $w,w' \in W$ as follows:
% \begin{equation}
%     \label{eq:porder}
% w \leq_{g(w)} w' :\iff \{p: p\in g(w) ~\&~ w' \in p\} \subseteq \{p: p\in  g(w) ~\&~ w \in p\}
% \end{equation}
% \todo{THIS NEED UPDATING SINCE g(w) CHANGED!}
\begin{equation}
    \label{eq:porder}
w \leq_{g(w)} w' :\iff \{A: g(w) \subseteq A ~\&~ w' \in A\} \subseteq \{A: g(w) \subseteq A ~\&~ w \in A\}
\end{equation}
The intended reading is that $w$ is at least as close to the ideal represented by $g(w)$ as $w'$ iff all the ideal propositions in $g(w)$ which are true in $w'$ are also true in $w$.
From this, Kratzer defines a the semantics for a list of different expressions, like `good possibility', `possibility' and so on.
For our purposes, it suffices to look at the following ordering on propositions:

The induced ordering on worlds $\leq_{g(w)}$ is `lifted' to an ordering on the level of propositions.
For a modal base $f$ and a ordering source $g$, for propositions $A,B$ and world $w$, we can define
\begin{equation}
    \label{eq:lifted}
    A \succeq_w B :\iff \forall b \in B_w \exists a \in A_w: a \leq_{g(w)} b,
\end{equation}
where $A_w = f(w) \cap A$, and $B_w$ analogously (cf. \cite[][p.~519]{holliday13_measur}).
In words, a proposition $A$ is as least as good a possibility as a proposition $B$ seen from a world $w$ iff for any world in $B$ epistemically accessible from $w$ there is a world in $A$ accessible from $w$ that is at least as close to the ideal picked out by $g(w)$.

In the next section, we describe how this ordering can be used to define a semantics for our propositional language.

\subsection{Formal Setup}

\begin{definition}
    A \emph{Kratzer preferential model} is a tuple $M = \langle W,f,g,V \rangle$ s.t.
    \begin{itemize}[nosep]
\renewcommand\labelitemi{--}
      \item $W$ is a non-empty set of possible worlds,
      \item $f,g$ are functions $f,g: W \rightarrow \wp(W)$,
      \item $V$ is a function $V: \text{Atom} \rightarrow \wp(W)$
    \end{itemize}
\end{definition}

Further, we want the induced order defined above also hold for $M$.
This is done via the following rather obvious definition:
Let $M = \langle W,f,g,V \rangle$ be a Kratzer preferential model.
For each $w \in W$ let $\succeq_w$ be the order on $\wp(W)$ defined in Equation \ref{eq:lifted}.
For $A,B \subseteq W$ we then define:   
\begin{equation}
    \label{eq:induced}
       A \succeq^{M,w} B \text{ iff } A \succeq_w B.
\end{equation}
    
\begin{definition}[Kratzer preferential semantics]
    \label{def:kratsem}
    Let $M = \langle M,f,g,V \rangle$ be a Kratzer preferential model, $w \in W$, $\varphi, \psi \in \L$ and let $\succeq^{M,w}$ be the induced order on $\wp(W)$ by Equation \ref{eq:induced}. We define 
    
    \begin{enumerate}[nosep]
      \item If $p \in \text{Atom}$, $M,w \models p$ iff $w \in V(p)$,
      \item $M,w \models \neg \varphi$ iff $M,w \not\models \varphi$
      \item $M,w \models (\varphi \land \psi)$ iff $M,w \models \varphi$ and $M,w \models \psi$,
      \item $\lb \varphi \rb^M = \{w' \in W : M,w' \models \varphi\}$,
      \item $M,w \models \varphi \succeq \psi$ iff $\lb \varphi \rb^M \succeq^{M,w} \lb \psi \rb^M$,
      \item $M,w \models \triangle \varphi$ iff $M,w \models \varphi \succeq \neg \varphi ~\&~ \neg (\neg \varphi \succeq \varphi)$,
    \end{enumerate}
\end{definition}

\subsection{Discussion}

How does Kratzer's account fare with respect to the adequacy conditions we defined?
The semantics validate conditions (V1-V5), which can be easily seen by applying the definitions.
For example, let us establish that Kratzer preferential semantics satisfy (V2), i.e.: 
\begin{center}
\begin{prooftree}
        \hypo{ \triangle (\varphi \land \psi)} \infer1[(V2)]{ \triangle \varphi \land \triangle \psi}
    \end{prooftree}
\end{center}
Suppose $M,w \models \triangle (\varphi \land \psi)$.
Then $\lb (\varphi \land \psi) \rb^M \succ^{M,w} \lb \neg(\varphi \land \psi)\rb^M$.
Let $y \in \lb \neg \varphi \rb^M$.
Then $y \in \lb \neg \varphi \lor \neg\psi \rb^M = \lb \neg(\varphi \land \psi)\rb^M$.
By assumption, there is an $x \in \lb (\varphi \land \psi) \rb^M$ such that $x \leq_{g(w)} y$.
But then $x \in \lb \varphi \rb^M$, too.
Hence $\lb \varphi \rb^M \succ^{M,w} \lb \neg \varphi \rb^M$ and thus $M,w \models \triangle \varphi$.
The case for $M,w \models \triangle \psi$ works analogously.

Kratzer's account does \emph{not} validate (V11-V12). \textcite[pp.~935, note~8]{yalcin10_probab_operat} gives a countermodel.
In addition, the account \emph{does} validate pattern (I1), as can easily be seen. This is already an unwelcome result. Additionally, (I1) together with (V5) entails (I2) \parencite[p.~922]{yalcin10_probab_operat}. That is, on this account, anything that is as probable as its negation is at least as likely as any tautology. This is, one can argue, a ruinous result.

Let's have a look at different accounts and see whether they fare better.

\section{Hamblin's Possibility Semantics}
Already in the 1950's, \textcite{hamblin59_modal_probab} developed a semantics for the \emph{probable}-operator based on what he called plausibility and more recently is referred to as possiblity. Possibility functions are quite similar to probability functions, but do not require finite additivity. They have been used to model an agent's belief state when the focus is on modeling the agent's surprise as well as ignorance towards some proposition. They are useful in modeling vagueness and have the advantage over probability functions that the possibility of the union of two propositions is uniquely determined by the marginal possibilities of the propositions, while probabilities only provide bounds unless the propositions are disjoint \parencite[][p.~45]{halpern03_reason_about_uncer}.

Why should possibilities be particularly appropriate in semantics? Hamblin does not present a clear argument, but refers to that the weighing of reasons is captured better in possibilities than in probabilities \parencite[][p.~240]{hamblin59_modal_probab}.
Yalcin gives his interpretation of Hamblin as that probabilities are too idealized, and possibilities provide a less restricted formal structure and are thus more appropriate to model everyday likelihood talk.
In any case, let us look at the formal setup and evaluate whether Hamblin's account holds up to our criteria. 
The account presented here is a reinterpretation based on \textcite{hamblin59_modal_probab}, \textcite[][pp.~42]{halpern03_reason_about_uncer} and \textcite{yalcin10_probab_operat} to fit the account into contemporary semantics.

\subsection{Formal Setup}
\noindent Let $W$ be a set. A boolean algebra on $W$ is a set $\F \subseteq \wp(W)$ such that
\begin{enumerate}[nosep]
  \item $\emptyset, W \in \F$,
  \item $A \in \F \Rightarrow \bar{A} \in \F$, and
  \item $A,B \in \F \Rightarrow A \cup B \in \F$.
\end{enumerate}

\noindent Let $\F$ be a boolean algebra. A possibility function is a function $p: \F \rightarrow [0,1]$ such that
\begin{enumerate}[nosep]
  \item $p(\emptyset) = 0$,
  \item $p(W) = 1$, and
  \item $p(U \cup V) = \max(p(U),p(V))$.
\end{enumerate}
\begin{definition}
    A \emph{Hamblin possibility model} is a tuple $M = \langle W,R,V,\mathcal{F},\Po\rangle$ such that
    \begin{itemize}[nosep]
        \renewcommand\labelitemi{--}
      \item $W$ is a set of possible worlds,
      \item $R$ is a binary relation on $W$,
        \item $V$ is a function $V: \text{Atom} \rightarrow \wp(W)$,
        \item $\mathcal{F}$ is a function assigning each world $w \in W$ a boolean algebra $\F_w$ on $R[w]$, and 
          \item $\Po$ is a function assigning each world $w \in W$ a possibility function \\ $p_w: \F_w \rightarrow [0,1]$.
    \end{itemize}
\end{definition}

\begin{definition}[Hamblin possiblity semantics]
    \label{def:hambsem}
    Let $M = \langle M,R,V,\mathcal{F},\Po \rangle$ be a Hamblin possiblity model, $w \in W$, $\varphi, \psi \in \L$. We define clauses 1-4 and 6 just like in Kratzer preferential semantics (Definition \ref{def:kratsem}). We define further:
    \begin{enumerate}[nosep]
  \setcounter{enumi}{4}
      \item $M,w \models \varphi \succeq \psi$ iff $p_w(\lb \varphi \rb^M) \geq p_w(\lb \psi \rb^M) $,
  \setcounter{enumi}{6}
    \end{enumerate}
\end{definition}
That the model contains a function assigning each world another function might seem formal overkill at first, but is just there such that the possiblity judgments are properly world-relative. When focusing on other aspects and e.g. developing a logic for this semantics, we might simplify our model in this regard, as \textcite{harrison-trainor17_prefer} do with preferential and probability models. 

\subsection{Discussion}

Hamblin's account satisfies (V1-V5), as can easily be seen. For example, let us briefly validate (V2) like for Kratzer's account.
Suppose $M,w \models \triangle(\varphi \land \psi)$, hence $p_w(\lb(\varphi \land \psi )\rb^M) > p_w(\lb \neg (\varphi \land \psi) \rb^M)$.
Because $p_w(\lb (\varphi \land \psi)\rb^M \cup {\lb\neg(\varphi \land \psi)\rb^M)} = 1$, by (2) and (3) of the definition of possibility functions, we have $p_w(\lb (\varphi \land \psi) \rb^M) = 1$.
If $W$ is finite, there is $w' \in \lb (\varphi \land \psi) \rb^M$ such that $p_w(\{w'\}) = 1$. Since $w'$ is in $\lb \varphi \rb^M$ and $\lb \psi \rb^M$, but there is no $w'' \in \lb \varphi \rb^M\setminus\lb \psi \rb^M$ or $\lb \psi \rb^M\setminus\lb \varphi \rb^M$ with $p_w(\{w''\}) = 1$ we have $p_w(\lb \varphi \rb^M) > p_w(\lb \neg \varphi \rb^M)$ and $p_w(\lb \psi \rb^M) > p_w(\lb \neg \psi \rb^M)$ and hence $M,w \models \triangle \varphi \land \triangle \psi$.

Further, Hamblin validates pattern (V11-V12) \parencite[][p.~926]{yalcin10_probab_operat}.
Like Kratzer, though, Hamblin also validates (I1) and (I2), but more egregiously, even (I3): 
\begin{center}
\begin{prooftree}
    \hypo{ \triangle \varphi } \infer1[(I3)]{ \varphi \succeq \psi}
    \end{prooftree}
\end{center}
To see this, note the fact we made use of above already: If $M,w \models \triangle \varphi$, then because $\lb \varphi \rb^M \cup \lb \neg \varphi \rb^M = W$, we have by (2) and (3) of the definition of possibility functions that $p_w(\lb \varphi \rb^M) = 1$ such that $M,w \models \varphi \succeq \psi$ for any $\psi \in \L$. This clearly runs counter to the idea of modeling the graded epistemic modal \emph{probable} in natural language. For other uses as in fuzzy logic \parencite[][p.~43]{halpern03_reason_about_uncer} this does of course say nothing.
\section{Yalcin's Probability Semantics}

Unsatisfied with the accounts described so far, Yalcin proposes to overcome the wariness towards probability functions as semantics for probability talk.
The semantics developed by \textcite{yalcin10_probab_operat} and very similarly by \textcite{lassiter10_gradab} is the absolutely straightforward application of probability measures towards our present problem, so obvious in fact that it strikes me as noteworthy that these seem to be among the earliest applications of probability functions to \emph{probable}-talk, as both Yalcin and Lassiter claim.
The following definitions are quite unsurprising, then.
\subsection{Formal Setup}

\noindent Let $\F$ be a boolean algebra. A probability function is a function $\mu: \F \rightarrow [0,1]$ such that
\begin{enumerate}[nosep]
  \item $\mu(\emptyset) = 0$ 
  \item $\mu(W) = 1$
  \item $\mu(U \cup V) = \mu(U) + \mu(V) \text{ if } U \cup V = \emptyset $
\end{enumerate}

\begin{definition}
    A \emph{Yalcin probability model} is a tuple $M = \langle W,R,V,\mathcal{F},\Pr\rangle$ such that
    \begin{itemize}[nosep]
        \renewcommand\labelitemi{--}
      \item $W$ is a set of possible worlds,
      \item $R$ is a binary relation on $W$,
        \item $V$ is a function $V: \text{Atom} \rightarrow \wp(W)$,
        \item $\mathcal{F}$ is a function assigning each world $w \in W$ a boolean algebra $\F_w$ on $R[w]$, and 
          \item $\Pr$ is a function assigning each world $w \in W$ a probability function \\ $\mu_w: \F_w \rightarrow [0,1]$.
    \end{itemize}
\end{definition}

\begin{definition}[Yalcin probability semantics]
    \label{def:yalcsem}
    Let $M = \langle M,R,V,\mathcal{F},\Pr \rangle$ be a Yalcin probability model, $w \in W$, $\varphi, \psi \in \L$. We define clauses 1-4 just like in Kratzer preferential models (Definition \ref{def:kratsem}). We define further:
    \begin{enumerate}[nosep]
  \setcounter{enumi}{4}
      \item $M,w \models \varphi \succeq \psi$ iff $\mu_w(\lb \varphi \rb^M) \geq \mu_w(\lb \psi \rb^M) $,
      \item $M,w \models \triangle \varphi$ iff $M,w \models \varphi \succ^{M,w} \neg \varphi$ iff $ \mu_w(\lb \varphi \rb^M) > .5$, and
    \end{enumerate}
\end{definition}

\subsection{Discussion}

Yalcin promotes his account as satisfying all criteria (V1-V5), (V11-V12), (I1-I3) we defined.
Let's look at some of them. 
To compare, the probability semantics validate V2: Suppose $M,w \models
\triangle(\varphi \land \psi)$. Then $\mu_w(\lb \varphi \land \psi\rb^M) > .5$.
Since marginal probabilities are bounded from below by joint probabilities of
the same variable, we have $\mu_w(\lb \varphi \rb^M) > .5$ and hence $M,w
\models \triangle \varphi$. The case for $\psi$ is analogous.
The semantics satisfy (I1) by invalidating the pattern.
\begin{center}
\begin{prooftree}
    \hypo{ \varphi \succeq \psi } \hypo{\varphi \succeq \chi}\infer2[(I1)]{ \varphi \succeq (\psi \lor \chi)}
    \end{prooftree}
\end{center}

\noindent To see this, fix a model $M$ and a world $w \in W$ (sub- and superscripts suppressed).
Let $\mu(\lb\varphi\rb) = .6;~ \lb\psi\rb \cap \lb\chi\rb=\emptyset;~ \mu(\lb\psi\rb) = \mu(\lb\chi\rb) = .4$. (that such a model exists can be easily verified).
Then $\mu(\lb \psi \rb \cup \lb \chi \rb) = .8$.
Thus $M,w \models \varphi \succeq \psi$ and $ M,w \models \varphi \succeq \chi$ but $M,w \not\models \varphi \succeq (\psi \lor \chi)$.

Given our criteria, then, the probabilistic model is the clear winner.
Does this
mean preferential accounts in general are inapt to model the
\emph{probable}-talk semantics?
Not quite, as we will see in the next section.

\section{Holliday and Icard's Preferential Semantics}

\textcite{holliday13_measur} are not yet ready to accept defeat for the preferentialist. First, they set out to generalize Yalcin's account and replace probability functions with so called qualitatively additive measures \parencite[][p.~522]{holliday13_measur}, which they show to also be adequate with respect to our criteria.\footnote{It is noteworthy that they describe (I2) as `If $\varphi \succeq \neg \varphi$ then $\varphi \succeq \psi$', instead of the present version (see above), since the resulting claims are different. It could be worthwile to investigate further where this discrepancy is coming from.}
They also briefly consider an ordering directly on the set of propositions instead of inducing the order from an order on worlds, and state that here, one can quite straightforwardly constrain the ordering such that it provides a semantic to satisfy the adequacy criteria \parencite[p.~924]{holliday13_measur}.
They dismiss it, however, because it is more desirable to not have a primitive order on propositions.
It seems they are themselves doubting this point and merely reiterate an established point.\footnote{This, too, seems worth investigating further.}

Finally, they propose an amendment to Kratzer's preferential semantics, intended to resolve the issues noted by Yalcin and Lassiter. With a seemingly small change on how the ordering is lifted from worlds to propositions, they manage to achieve a host of desired effects. For example, while Kratzer's lifting creates a total ordering on propositions if the order on worlds is total, not so for Holliday and Icard's lifting (cf. \parencite[][p.~525]{holliday13_measur} for a nice extended argument for why this is desirable). We describe the precise change in the next section, and have a look at the results concerning our criteria after that.

\subsection{Formal Setup}
\begin{definition}
    A Holliday and Icard preferential model is a tuple $M = {\langle
      W,R,V,\succeq\rangle}$ such that
    \begin{itemize}[nosep]
        \renewcommand\labelitemi{--}
      \item $W$ is a set of possible worlds,
      \item $R$ is a binary relation on $W$,
      \item $V$ is a function $V: \text{ Atom } \rightarrow \wp(W)$, and
      \item $\succeq$ is a function assigning each world $w \in W$ a preorder
        (reflexive and transitive binary relation) $\geq^{M,w}$ on $W$. 
    \end{itemize}
\end{definition}
For any proposition $P \subseteq W$, $P_w := P \cap R[w]$. Further, for two proposition $A,B \subseteq W$ we define
\begin{multline}
    \label{eq:injection}A \succeq^{M,w} B :\iff \text{ there exists an injective function } f: B_w \rightarrow A_w \\ \text{ such that } f(x) \geq^{M,w} x \text{ for all } x \in B_w.
\end{multline}

\begin{definition}[Holliday and Icard preferential semantics] Let $M = {\langle
      W,R,V,\succeq \rangle}$, $w \in W, \varphi,\psi \in \L$ and let $\succeq^{M,w}$ be the order on $W$ defined by Equation \ref{eq:injection}. We the define clauses 1-6 just like in Kratzer preferential semantics (Definition \ref{def:kratsem}).
\end{definition}

Holliday and Icard's semantics is very similar to Kratzer's.
In particular, we can associate the modal base and ordering source of Kratzer's with the binary relation $R$ and the $\geq$-function.
Just as the modal base determines for each world a set of epistemically accessible worlds, so does $R[w]$.
And just as the ordering source induces an order on propositions, so does $\geq$.
For our present purposes, both formulations are equivalent even.
Holliday and Icard make a crucial change to how the order is induced, however.
To see this, note that Equation \ref{eq:lifted} can be rewritten as an equivalent definition:

For a modal base $f$ and a ordering source $g$, for propositions $A,B$ and world $w$, define
\begin{multline}
    \tag{2*}
    \label{eq:lifted-star}
    A \succeq_w B :\iff \text{ there exists a function } f: B_w \rightarrow A_w \\ \text{ such that } f(x) \leq_{g(w)} x \text{ for all } x \in B_w.
\end{multline}
Compared to Equation \ref{eq:injection}, besides the inverted direction of $\geq$ which is just due to how the notions were defined, the difference is that Holliday and Icard require an injective function assigning each element of $B$ a unique element of $A$ such that the element of $A$ is more plausible, normal etc.
This simple change has interesting consequences, as we will see in the next section. 
\subsection{Discussion}

Holliday and Icard promote their account as satisfying all adequacy criteria (V1-V5), (V11-V12), (I1-I3).
Let's have a closer look.
First, as an example, the account again satisfies (V2).
To show this, we can take the argument from Kratzer's semantics, and amend it slightly such that Holliday and Icard's condition (\ref{eq:injection}) is satisfied.  
We will spare the reader such repetitious reading. 

We can instead employ the model Holliday and Icard provide to construct a countermodel to (I1). Recall that Kratzer's account validated this pattern. We adopt slightly sloppy notation to improve readability.  

Fix a model $M$ and a world $w$. Let $W = \{abcd\}$, ordered by $M$ like so: $a > b > c > d$. Let $A = \{ab\}, B = \{bc\}, C = \{cd\}$. It's easy to see that by (\ref{eq:injection}), we have $A \succ B \succ C$. However, we do not have $A \succ (B \cup C)$, since we can't construct an injection like required. Hence, (I1) is not validated. A clear improvement over Kratzer's semantics.

\section{Conclusion}
Maybe make an overview table from org mode
\todo{
  \begin{itemize}
    \item Write Conclusion and further studies
  \end{itemize}
}
What about the $\epsilon$-semantics (Adams, Pearl)? 

\nocite{hamblin59_modal_probab,holliday13_measur,harrison-trainor17_prefer,kratzer91_modal,lassiter10_gradab,yalcin10_probab_operat,kratzer98_seman}
 \printbibliography
\end{document}